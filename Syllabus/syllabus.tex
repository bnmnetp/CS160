\documentclass[11pt,twocolumn]{article}
\usepackage[mathletters]{ucs}
\usepackage[utf8x]{inputenc}
\usepackage[text={6in,9.5in},centering]{geometry}

\usepackage{ifpdf}
\usepackage{graphicx}
\usepackage{sectsty}
\allsectionsfont{\sffamily}
\setlength{\parindent}{0pt}
%\setlength{\parskip}{6pt plus 2pt minus 1pt}

\usepackage[breaklinks=true]{hyperref}
\title{Data Structures}
\setcounter{secnumdepth}{0}
\author{}
\begin{document}

\begin{center}
    {\sffamily\LARGE\bfseries CS-160 Data Structures}
\end{center}

\section*{Instructor}
Brad Miller \\
Olin, 321 \\
email:  bmiller@luther.edu \\
Skype:  bonelake \\
Google+ \\

\section*{Office Hours}
Monday -- Friday: (1:30--2:30) \\
Other times by appointment, drop-in, or virtual.  Really! I'm here to help you, so stop in.

\section*{Text Book}

We'll be using the book \textit{Problem Solving with Algorithms and Data
  Structures using Python}, by Brad Miller and David Ranum.  With the blessing of
our publisher.  Its available on http://interactivepython.org

\section*{Goals}

\begin{itemize}
    \item To continue to improve your problem solving skills
    \item To increase your comfort with writing Python programs
    \item To become more "pythonic" in your coding practices
    \item To understand how Python implements common data structures and the tradeoffs incurred by those decisions.
    \item To learn to recognize common patterns in problem solving
    \item To learn how to critically evaluate algorithms
\end{itemize}

\section*{Course Outline}
\begin{enumerate}
    \item Introduction

\begin{enumerate}
    \item Review of Python Basics
    \item More on defining our own classes
    \item Inheritance
\end{enumerate}
    \item Algorithm Analysis

\begin{enumerate}
    \item Big O analysis
    \item Experimental analysis
    \item Python data structures performance

\end{enumerate}
    \item Basic Data Structures

\begin{enumerate}
    \item Stacks
    \item Queues
    \item Dequeues
    \item Linked Lists

\end{enumerate}
    \item Recursion

\begin{enumerate}
    \item The 3 laws of Recursion
    \item Easy Recursive Problems
    \item Hard Recursive Problems
    \item Dynamic Programming -- When Recursion takes too long

\end{enumerate}
    \item Searching and Sorting

\begin{enumerate}
    \item Linear search
    \item Binary search
    \item Sort Algorithms
    \item Bubble Sort
    \item Insertion Sort
    \item Merge Sort
    \item Radix sort
    \item Quicksort

\end{enumerate}
    \item Trees

\begin{enumerate}
    \item Binary Trees
    \item Heaps
    \item Balanced Binary Trees

\end{enumerate}
    \item Graphs

\begin{enumerate}
    \item Breadth First Search  (The famous Kevin Bacon problem)
    \item Depth First Search
    \item Topological Sorting
    \item Strongly Connected Components
    \item Shortest Path Problems
    \item Dijkstra's Algorithm
    \item Prims algorithm

\end{enumerate}
\end{enumerate}
\section*{Class Requirements:}

\subsection*{Participation}

``The teacher's job is to design learning experiences, not primarily to impart information''  Although there will be some lecture to this course a substantial amount of time will be spent solving problems and doing things.  Please, read the assigned materials and come to class with questions.  It is fine to be confused. You are not going to understand everything the first time you read it or hear it or do it.

\subsection*{Homework}

You will have at least weekly homework assignments, sometimes more frequently.  You need to hand them in on-time, and I, in turn, will get them graded back to you within a week.

% subsection final_project (end)
\subsection*{Grading}


\begin{verbatim}
50%   Labs and Homework
40%   Exams  (likely 3 exams)
10%   Class participation
\end{verbatim}

The grading scale is:

\begin{verbatim}
90 -- 100   A
80 -- 89    B
70 -- 79    C
60 -- 69    D
0 --  59    F
\end{verbatim}


\end{document}
